\chapter{Introdução}\label{chapter:introducao}

Computação em nuvem é um termo que surgiu na década de 1960 com os sistemas de compartilhamento de tempo. Atualmente, define-se computação em nuvem como a entrega de recursos de TI através da Internet e sob demanda com definição de preço conforme o uso (Amazon AWS, 2022). Dessa forma, empresas e usuários de tecnologia conseguem expandir seus negócios de maneira rápida e estratégica sem ter que se preocupar com uma infraestrutura própria de TI. Algumas empresas que fornecem serviço de computação em nuvem são: AWS, Google Cloud Platform, Microsoft Azure e IBM Cloud.

Dessa forma, a nuvem oferece agilidade, elasticidade e economia de custos para organizações de todos os tipos, portes e setores. Esse tipo de tecnologia permite a geração rápida de recursos de acordo com a evolução das necessidades empresariais. Assim, empresas que fazem uso de computação em nuvem não correm o risco de provisionar recursos em excesso para absorver picos de atividades empresariais futuras. Por fim, a nuvem permite a troca de despesas fixas, como \textit{datacenters}, segurança de dados e servidores públicos, por despesas variáveis que dependem somente do que foi consumido.

% Atualmente o termo "sistemas embarcados" não está muito presente no cotidiano brasileiro, mas essa tecnologia é responsável por fornecer equipamentos inteligentes como semáforos de trânsito, relógios, respiradores mecânicos, roteadores e aparelhos de ar-condicionado, por exemplo. Em termos simples, pode-se definir um sistema embarcado como um dispositivo controlado por um computador encapsulado. Ou seja, um sistema embarcado é na verdade um sistema microprocessado. Dentre as principais vantagens de sistemas embarcados, destaca-se o baixo custo, a eficiência e a facilidade de programação (NORLETO, 2020).

% O primeiro sistema embarcado é o AGC (Apollo Guidance Computer). Ele foi desenvolvido nos EUA por Charles Stark Draper do MIT em 1966 (EMBARCADOS, 2014). Desde então, os sistemas embarcados ficaram mais acessíveis, mais velozes e mais compactos.

% Com o passar dos anos, novas tecnologias foram adicionadas à sistemas embarcados, como por exemplo o Wifi, o Bluetooth etc. Na última década foi observado a ascensão de assistentes virtuais como a Alexa e o Google Assistente. A Alexa foi criada em 2014 e apareceu pela primeira vez como parte das caixas de som. Hoje a Alexa, a partir de comandos de voz, pode realizar pesquisas, mandar executar uma lista de músicas, disparar um alarme etc.(VIGLIAROLO, 2017). Segundo o site oficial da Alexa (2022), o serviço permite a conexão com dispositivos, efetuar comandos por voz, interpretá-los e tomar uma ação correspondente. Isso tudo acontece por meio da AWS.

Já o termo Internet das Coisas, ou simplesmente IoT, surgiu em 1999 com um sistema de sensores omnipresentes conectados à internet. A tecnologia ganhou popularidade e se expande com o crescimento de tecnologias de conectividade, computação em nuvem e aprendizado de máquina, por exemplo. Uma definição simplificada de IoT é: uma rede de itens conectados à internet. A principal vantagem dessa tecnologia é a não intervenção humana, o que gera um aumento na produtividade de tarefas repetitivas e escalabilidade.

Outra tecnologia recente são as assistentes virtuais inteligentes. Uma assistente virtual inteligente é um agente de software que pode realizar tarefas ou serviços para um indivíduo (Wikipedia, 2022). Algumas informações que as assistentes são capazes de acessar e rapidamente leva-las ao usuário, em linguagem natural, são: notícias, condições meteorológicas e de trânsito, agenda do usuário, entre outras. Alguns exemplos disponíveis no mercado são a Alexa, Microsoft Cortana, Siri e Google Assistant.

Em 25 de setembro de 2019, o Alexa e o Google Assistant puderam ajudar seus usuários a se candidatarem a empregos no McDonald's usando serviços de reconhecimento de voz. É o primeiro serviço de emprego do mundo usando o serviço de comando de voz. A Amazon anunciou em 25 de setembro de 2019 que Alexa em breve poderá imitar vozes de celebridades, incluindo Samuel L. Jackson, custando US \$ 0,99 para cada voz (Wikipedia, 2022).

Dessa forma, uma tendência das tecnologias focadas na experiência do usuário, como a IoT, é o suporte às assistentes virtuais. Diante do que foi dito, este trabalho propõe criar um prototipo de um nó IoT integrado à assistente de voz Alexa e serviços em nuvem.

\section{Estrutura}
O trabalho será apresentado em cinco capítulos. O \autoref{chapter:introducao} apresenta a introdução, contexto histórico e a motivação do projeto. O \autoref{chapter:referencial_teorico} apresenta o referencial teórico e trabalhos correlatos. O \autoref{chapter:metodologia} contém o desenvolvimento do trabalho, apresentando requisitos funcionais e não-funcionais, ferramentas utilizadas e o processo de desenvolvimento. O \autoref{chapter:resultados_e_discussoes} faz a análise dos resultados. O \autoref{chapter:conclusoes}, por fim, conclui o trabalho, listando dificuldades enfrentadas, vantagens, desvantagens e as sugestões para trabalhos futuros. O projeto conta também com o \autoref{chapter:apendice}, que é um capítulo extra de apêndice.
