\chapter{Introdução}\label{chapter:introducao}

REFACTORING...

% Atualmente, observa-se uma ascensão do termo "computação em nuvem" no cotidiano de empresas e no vocabulário popular. Dessa forma, muitas pessoas definem nuvem como aplicativos como o Google Drive e o Dropbox. As nuvens, contudo, são compostas por vários recursos de computação, desde os próprios computadores até redes, armazenamento, bancos de dados e os recursos ao seu redor. Ou seja, tudo o que normalmente é necessário para construir o equivalente a uma sala de servidores, ou mesmo um data center completo, está prontamente disponível, configurado e em execução.

% . Contudo, essa tecnologia teve suas origens na década de 1960. Assim, os sistemas de compartilhamento de tempo foram os primeiros a oferecer um recurso compartilhado ao programador.

% Atualmente o termo "sistemas embarcados" não está muito presente no cotidiano brasileiro, mas essa tecnologia é responsável por fornecer equipamentos inteligentes como semáforos de trânsito, relógios, respiradores mecânicos, roteadores e aparelhos de ar-condicionado, por exemplo. Em termos simples, pode-se definir um sistema embarcado como um dispositivo controlado por um computador encapsulado. Ou seja, um sistema embarcado é na verdade um sistema microprocessado. Dentre as principais vantagens de sistemas embarcados, destaca-se o baixo custo, a eficiência e a facilidade de programação (NORLETO, 2020).

% O primeiro sistema embarcado é o AGC (Apollo Guidance Computer). Ele foi desenvolvido nos EUA por Charles Stark Draper do MIT em 1966 (EMBARCADOS, 2014). Desde então, os sistemas embarcados ficaram mais acessíveis, mais velozes e mais compactos.

% Com o passar dos anos, novas tecnologias foram adicionadas à sistemas embarcados, como por exemplo o Wifi, o Bluetooth etc. Na última década foi observado a ascensão de assistentes virtuais como a Alexa e o Google Assistente. A Alexa foi criada em 2014 e apareceu pela primeira vez como parte das caixas de som. Hoje a Alexa, a partir de comandos de voz, pode realizar pesquisas, mandar executar uma lista de músicas, disparar um alarme etc.(VIGLIAROLO, 2017). Segundo o site oficial da Alexa (2022), o serviço permite a conexão com dispositivos, efetuar comandos por voz, interpretá-los e tomar uma ação correspondente. Isso tudo acontece por meio da AWS.

% Em 25 de setembro de 2019, o Alexa e o Google Assistant puderam ajudar seus usuários a se candidatarem a empregos no McDonald's usando serviços de reconhecimento de voz. É o primeiro serviço de emprego do mundo usando o serviço de comando de voz. A Amazon anunciou em 25 de setembro de 2019 que Alexa em breve poderá imitar vozes de celebridades, incluindo Samuel L. Jackson, custando US \$ 0,99 para cada voz.

% Visto que as tecnologias Amazon AWS e Alexa são recente, ainda não se observa uma variedade de dispositivos, não comercializados pela Amazon e pela Google, que fazem o uso de linguagem natural. Com a baixa oferta de produtos com essa tecnologia, produtos simples - como interruptores - estão sendo comercializados por preços não acessíveis ao mercado global.

% Diante do que foi dito, este trabalho se propõe criar um prototipo de baixo custo integrado à assistente de voz Alexa e serviços em nuvem.

\section{Estrutura}
O trabalho será apresentado em cinco capítulos. O \autoref{chapter:introducao} apresenta a introdução, contexto histórico e a motivação do projeto. O \autoref{chapter:referencial_teorico} apresenta o referencial teórico e trabalhos correlatos. O \autoref{chapter:metodologia} contém o desenvolvimento do trabalho, apresentando requisitos funcionais e não-funcionais, ferramentas utilizadas e o processo de desenvolvimento. O \autoref{chapter:resultados_e_discussoes} faz a análise dos resultados. O \autoref{chapter:conclusoes}, por fim, conclui o trabalho, listando dificuldades enfrentadas, vantagens, desvantagens e as sugestões para trabalhos futuros. O projeto conta também com o \autoref{chapter:apendice}, que é um capítulo extra de apêndice.
