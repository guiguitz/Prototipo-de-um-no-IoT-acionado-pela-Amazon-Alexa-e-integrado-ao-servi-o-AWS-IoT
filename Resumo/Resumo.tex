% -----------------------------------------------------------------------------
% Resumo
% -----------------------------------------------------------------------------
\begin{resumo}
    O trabalho tem como escopo a criação de um protótipo de nó IoT acionado pela Alexa através da computação em nuvem da AWS (Amazon Web Services). O protótipo utilizou o kit de desenvolvimento B-L475E-IOT01A2, da STMicroelectronics, que se conecta na Internet por meio de uma rede sem fio. A integração do nó IoT com a Alexa acontece via protocolo MQTT (MQ Telemetry Transport), AWS Lambda e AWS IoT. Para a criação de comandos específicos para o protótipo, compreendidos pela Alexa, uma Alexa Skill foi desenvolvida utilizando o ASK (Alexa Skills Kit). Todo o back-end da aplicação foi desenvolvido em Python, é hospedado no AWS Lambda e pode ser descrito por um diagrama de estados. Como resultado, o usuário é capaz de acionar a alternância do estado de um LED do dispositivo através das frases ``set led on'', ``set led off'', ``turn led on'' e ``set led off''.

    Palavras-chave: ASK, AWS IoT, AWS Lambda, Amazon Alexa, B-L475E-IOT01A2, Computação em Nuvem, IoT, MQTT, ask-sdk.
\end{resumo}
